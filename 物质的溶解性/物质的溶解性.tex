\documentclass[8px,a4paper]{article}
\usepackage[UTF8]{ctex}
\usepackage[T1]{fontenc}
\usepackage[left=5mm, right=5mm, top=5mm, bottom=5mm]{geometry}
\usepackage[version=4]{mhchem}
\usepackage{chemfig}
\newcommand*{\ncn}[2]{\nl \ce{#1} #2}
\newcommand*{\nc}[1]{\nl \ce{#1}}
\newcommand*{\nl}{\indent\indent}
\title{物质的溶解性}
\author{[n-KaN-f]}
\begin{document}
	\section{物质的溶解性}
	本章名言警句:记少不记多
	\subsection{\ce{K+, Na+, NH4+, NO3- ,HCO3- ,HSO3- ,CH3COO-}对应的物质基本上都溶于水}
	例外: \ce{CoC2O4 v} ( \ce{C2O4^2-}定向沉 \ce{CO^2+} )
	
	
	\emph{补充: \ce{CO3^2-}沉 \ce{ Li+, Ni^2+}; }
	
	
	\emph{补充: \ce{SO4^2-}沉 \ce{ Pb^2+}; \ce{(CH3COO)2Pb} 溶于水,弱电解质。}
	
	
	\emph{补充: 为什么 \ce{C2O4^2-} 的盐大多易溶于水?}
	
	
	\chemfig{C(=[2,0.7]O)(-[0,0.7]C(=[2,0.7]O)(-[7,0.7]C))(-[5, 0.7]C)}
	\ce{<-} 草酸的结构,羧基容易与金属阳离子形成配位键,故易溶于水;且构成五元环的螯合物,很稳定。
	\subsection{氯化物 \ce{Cl-} 除 \ce{Ag+, Cu+}\emph{(亚铜离子)} 之外,全部溶于水}
	氯化物三大性质: 
	
	
	1. 除了 IA, IIA, 绝大多数为共价化合物。(IIA中的\ce{BeCl2}也是共价化合物)
	
	
	2. 熔沸点低,易升华,易堵塞导管(解决方法为换粗导管)。
	
	
	3. 易水解,需左右隔水。
	
	\indent\indent \ce{SOCl2 ->T[水] H2SO3 + HCl}
	\indent\indent \ce{POCl3 ->T[水] H3PO4 + HCl}
	\indent\indent \ce{TiOCl2 ->T[水] TiOOH + HCl}
	
	
	实验应用:
	
	\nl \ce{CuCl2 ->T[一直加热] Cu(OH)2 ->T[煅烧] CuO} \emph{氯化物易水解,所以在加热的过程中\ce{CuCl2}水解程度加大,最终变成\ce{Cu(OH)2}}
	
	\nl 配制\ce{FeCl3},溶于浓盐酸(防水解), \emph{\ce{SOCl2}也能防水解(除水)}
	
	\nl \ce{MgCl2*6H2O ->T[在HCl的氛围里加热脱水] MgCl2}
	
	
	\emph{补充:}
	
	\nl \ce{CaCl2} 不能干燥 \ce{CH3CH2OH, NH3, SO3}
	
	\nl \ce{CaCl2 + 2NH3 -> CaCl2*2NH3}
	\nl \ce{CaCl2 + 2CH3CH2OH -> CaCl2*2CH3CH2OH}
	
	\nl \ce{CaCl2 + SO3 -> CaSO4 v + HCl}
	
	
	\subsection{\ce{SO4^2-} 除 (\ce{Ba^2+, Pb^2+, })(\emph{不溶}) 和 (\ce{Ag+, Ca^2+})(\emph{微溶}) 之外都溶于水}
	\emph{补充:微溶表示除不尽,之后的流程需要再除一次,但前面过滤的滤渣需考虑微溶。}
	
	
	补充:硫酸盐=X矾
	
	\nl \ce{FeSO4*7H2O} 绿矾; \ncn{CuSO4*5H2O}{胆矾}; \ncn{KAl(SO4)2*12H2O}{明矾};
	
	\ncn{Na2SO4*10H2O}{芒硝(一种泻药)}; \ncn{ZnSO4*7H2O}{皓矾};
	
	
	补充:生石膏\emph{(打石膏默认)}\ce{CaSO4*2H2O}; \indent 熟石膏 \ce{CaSO4*0.5H2O}
	
	
	[待写:结晶水的问题]
	
	
	[待写:\ce{S2O3^2-}]
	
	
	\subsection{\ce{CO3^2-} 除 \ce{Na+, K+, NH4+} 之外,基本上都是沉淀}
	特征:\ce{CaCO3 ->T[$\Delta$] CaO + CO2 ^}
	
	\nl 不溶于水的碳酸盐易受热分解 \nl 原因:熔融状态下 \ce{CO3^2- -> CO2 ^ + O^2-}
	
	
	讨论分解温度高低:
	
	\nl 温度高 \ce{CaCO3 ->T[$\Delta$] CaO + CO2 ^}
	
	\nl 温度低 \ce{MgCO3 ->T[$\Delta$] MgO + CO2 ^}
	
	\nl 原因:金属阳离子夺取 \ce{CO3^2-}中 \ce{O^2-}的能力越强,分解温度越低。
	
	\nl \ce{Mg^2+}强的原因:$F=k\frac{Q_1Q_2}{r^2}$ 其中  \ce{r(Mg^2+)}$<$\ce{r(Ca^2+)}则 $F_{\ce{Mg^2+}}>F_{\ce{Ca^2+}}$, $T_{\ce{Mg^2+}}<T_{\ce{Ca^2+}}$
	
	
	讨论热稳定性:
	
	\nl 不稳定性排序:\ce{H2CO3}$>$碳酸氢盐$>$不溶性碳酸盐$>$可溶性碳酸盐
	
	\nl \emph{补充:是否溶于水和极性有关}
	
	\nl \emph{补充:碳酸氢盐中,\ce{Ca(HCO3)2}更易分解,因为其极性更强}
	
	
	\subsection{\ce{OH-} 除 \ce{Na+, K+, NH4+, Ba^2+} 之外几乎都是沉淀}
	\emph{补充:\ce{Ca(OH)2}的溶解度随温度升高而下降。(可用于判断反应是吸热还是放热)}
	
	
	\emph{补充:弱碱都不稳定,受热易分解}
	
	\nc{2Al(OH)3 ->T[$\Delta$] Al2O3 + 3H2O} \nc{Ni(OH)2 ->T[$\Delta$] NiO + H2O} \nc{TiO(OH)2 ->T[$\Delta$] TiO2 + H2O}
	
	\nc{Co(OH)3 ->T[$\Delta$] Co2O3 + H2O}
	
	
	\emph{补充:}\ce{NiOOH, MnOOH, Co(OH)2, Co(OH)3, Ni(OH)2, Ni(OH)3} \emph{都是沉淀,除杂方法:调$pH$值}
	
	
	补充:氨水(\ce{NH3*H2O})和\ce{Mg(OH)2}的互制
	
	\nc{2NH4Cl + MgO ->T[$\Delta$] 2NH3 ^ + MgCl2}
	
	
	[待写:全空体系]
	
	
	利用$K_{sp}$除杂: \ce{Mn^2+(Ni^2+)} 加入\ce{MnS}:
	
	\nl $K_{sp}(\ce{MnS})>K_{sp}(\ce{NiS})$ 实现沉淀转化:\ce{MnS + Ni^2+ -> NiS + Mn^2+}
	
	
	\subsection{硫化物(\ce{S^2-})}
\end{document}